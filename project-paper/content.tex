% status: 0
% chapter: TBD

\title{Abstract File System}


\author{Swarnima Sowani}
\affiliation{%
  \institution{Indiana University}
  \streetaddress{Smith Research Center}
  \city{Bloomington} 
  \state{IN} 
  \postcode{47408}
}
\email{shsowani@iu.edu}

\author{Gregor von Laszewski}
\affiliation{%
  \institution{Indiana University}
  \streetaddress{Smith Research Center}
  \city{Bloomington} 
  \state{IN} 
  \postcode{47408}
  \country{USA}}
\email{laszewski@gmail.com}


% The default list of authors is too long for headers}
\renewcommand{\shortauthors}{G. v. Laszewski}


\begin{abstract}
bstract File System is a web services package that implements and abstracts
the underlying storage services and provides a uniform web services APIs for
users to do file operations like, retrieving, storing, removing  etc. The
services are provided to support storage engines such as Amazon Simple Storage
Service, google drive and virtual machine. This web service package can be
used by different big data applications client to perform file operations with
consolidated data view. 


\end{abstract}

\keywords{hid-420, File system, abstraction, Amazon S3, Google drive, virtual machine}


\maketitle

\section{Introduction}


File system operations is an integral part of the operating system and a
quintessential part of most of the applications that deals with storing,
retrieving or reading the files. Now a days it is likely possible to have data
stored at different locations with different storage systems. Managing all this
data together is a difficult task.

There are numerous storage providers who facilitates their customers to store,
read, write, retrieve and other sorts of file operations, however integrating
with each one of them is a tedious tasks since, understanding and implementing
APIs for each service is a time consuming tasks.

That is why abstract file system comes into picture. Abstract file system
provides a layer of abstraction over the underlying file system so that the
users or customers can integrate storage services as file systems like 
Amazon S3, Google Drive and other services without even 
knowing their real implementation.


The project goals are
\begin{enumerate}
\item	Create a service to connect to some virtual system and perform file operations.
\item	Create a service to access Google drive data~\cite{hid-sp18-420-google-drive}.
\item	Create a service to access data on S3.
\item	Create a service which will internally call above 
3 services to return the consolidated view of file system. 
\end{enumerate}

\section{Technology Used}

\section{Supported Storage Engines}
Abstract file system is currently supporting three types of storage engines
which are Amazon S3, Google drive and any virtual machine.

\subsection{Amazon S3}

Amazon S3 stands for Simple Storage Service. It is the most popular storage
service provided by Amazon Web Service. It provides a highly scalable,
reliable, and low latency data storage infrastructure at low costs. Amazon S3
can be used to store and retrieve data of any kind and any amount from anywhere.

Amazon S3 is known for its durability and stability. Amazon S3 SLAs claims to
provide 99.999999999\% durability for files stored in its durable 
storage~\cite{hid-sp18-420-amazon-S3}.

Amazon S3 is popularly an object storage, where each file is treated as an
object. Amazon S3 claims no cap on the amount of data that can be stored, that
is why companies who needs scale on the go prefer to choose Amazon S3 as their
file or object storage.

Abstract File System encashes on the scalability and simplicity Amazon S3
provides by abstracting the underlying mechanism and providing a way to users
to store their data into the durable S3 storage.\\


\textbf{File Operations Supported by Abstract File System}
\\

\begin{itemize}
    \item   Listing of files inside a specified bucket (bucket logical
segregation of files)
    \item       Retrieving a specific file
    \item       Storing file in  the specified bucket
    \item       Removing the file from the bucket
\end{itemize}

\textbf{How to enable Amazon S3 as a storage in Abstract File System}\\
In order to use Amazon S3 as a storage system, first user needs to have an
account with AWS system.
\textbf{Steps to create new root account with AWS}\\
\begin{enumerate}
    \item Open https://aws.amazon.com/ and click Create New Account.
    \item Give details such as email address, password and user name.
    \item Contact details: Give all specified contact details.
    \item Payment Information: Give credit card or debit card details for user
verification by AWS side.
    \item       Phone verification: AWS will make a call on the given number
and give a 4 digit code to verify your phone number.
    \item Select a support plan and Continue.
\end{enumerate}
\\
\textbf{Steps to create new user in AWS account and giving programmatic access}
\\
\begin{enumerate}
    \item Login to AWS account.
    \item Go to Services and select IAM from drop down. 
	IAM stands for Access and Identity Management.
    \item Inside IAM resources, click on User.
    \item This will show the list of users. Click on Add User to add a new user.
    \item Provide user name and select programmatic access.
    \item On next page, provide optional permissions to user such as S3 Full
access and similar.
    \item On next page, review all settings and click Next.
    \item Access key and secret key are displayed here. Save it somewhere safe.
Key file can be downloaded here in \.csv format.
\end{enumerate}
\\

\textbf{To use Amazon S3 as a storage engine within Abstract File System}, AWS
credentials must be provided in the config.yaml file.
Abstract file system requires user to provide 3 parameters to start using S3
service.
\begin{enumerate}
 \item  Access Key
 \item  Secret Key
 \item  AWS region
 \item  Bucket name
\end{enumerate}

Access key and secret key are required for the programmatic authentication for
using S3 services. These keys are provided at the time of account creation of
user.

While creating a new bucket, an AWS region is assigned to that bucket. Though
the buckets are globally unique and can be accessible from anywhere, they are
placed in specific AWS region.
Amazon S3 stores data within resources called bucket. Buckets are logical
segregation of files. Bucket name is configurable from the config.yaml file
which specifies the location where all file operations can be performed.


\begin{acks}

  The author would like to thank Dr.~Gregor~von~Laszewski for his
  support and suggestions to write this paper.

\end{acks}

\bibliographystyle{ACM-Reference-Format}
\bibliography{report} 

