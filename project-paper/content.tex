% status: 0
% chapter: TBD

\title{Abstract File System}


\author{Swarnima Sowani}
\affiliation{%
  \institution{Indiana University}
  \streetaddress{Smith Research Center}
  \city{Bloomington} 
  \state{IN} 
  \postcode{47408}
}
\email{shsowani@iu.edu}

\author{Gregor von Laszewski}
\affiliation{%
  \institution{Indiana University}
  \streetaddress{Smith Research Center}
  \city{Bloomington} 
  \state{IN} 
  \postcode{47408}
  \country{USA}}
\email{laszewski@gmail.com}


% The default list of authors is too long for headers}
\renewcommand{\shortauthors}{G. v. Laszewski}


\begin{abstract}
Abstract file system project is based on developing a rest services that
provides abstraction for file system.
This project aims at providing a common way to access entire file system. 


\end{abstract}

\keywords{hid-420, File system, abstraction}


\maketitle

\section{Introduction}

Abstract file system provides rest APIs to access different storage systems. Now
a days it is likely possible to have data stored at different locations such as
personal laptop, desktop or google drive. With the abstract file system, it will
be easy to access any storage resources in a consolidated way. The idea here is
to create different services with one service per resource. Here, resources are
local system, remote system, google drive and S3. Once these services are created
for individual resource, a single service can be created which will internally call all
the developed services in order to provide a consolidated view of data.

The project goals are
\begin{enumerate}
\item	Check an authentication mechanism to access the local system with REST API developed.
\item	Create a service which will work for the given directory on local machine.
\item	Create a service to access google drive data~\cite{hid-sp18-420-amazon-S3-FAQ}.
\item	Create a service to access data on S3.
\item	Create a service which will internally call individual service for local system, 
google drive and S3 to return a consolidated view of file system. 
\end{enumerate}



\begin{acks}

  The author would like to thank Dr.~Gregor~von~Laszewski for his
  support and suggestions to write this paper.

\end{acks}

\bibliographystyle{ACM-Reference-Format}
\bibliography{report} 

